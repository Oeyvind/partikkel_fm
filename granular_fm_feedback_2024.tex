
%%%%%%%%%%%%%%%%%%%% file icsc2017_template.tex %%%%%%%%%%%%%%%%%%%%%
%
% This is the LaTeX source for the instructions to authors using
% the LaTeX document class 'llncs.cls' for contributions to
% the Lecture Notes in Computer Sciences series.
% http://www.springer.com/lncs       Springer Heidelberg 2006/05/04
%
% It may be used as a template for your own input - copy it
% to a new file with a new name and use it as the basis
% for your article.
%
% NB: the document class 'llncs' has its own and detailed documentation, see
% ftp://ftp.springer.de/data/pubftp/pub/tex/latex/llncs/latex2e/llncsdoc.pdf
%
%%%%%%%%%%%%%%%%%%%%%%%%%%%%%%%%%%%%%%%%%%%%%%%%%%%%%%%%%%%%%%%%%%%


\documentclass[runningheads,a4paper]{llncs}

\usepackage{amssymb}
\setcounter{tocdepth}{3}
\usepackage{graphicx}
% \usepackage{url}
\usepackage{hyperref}
\hypersetup{hidelinks}

\usepackage{tikz}
\usetikzlibrary{shapes, arrows, bending, positioning}

\tikzstyle{process} = [rectangle, rounded corners, minimum width=3cm, minimum height=0.8cm, text centered, draw=black, fill=red!30]
\tikzstyle{data}=[trapezium, draw=black, text centered, trapezium left angle=60, trapezium right angle=120, minimum height=2em,fill=orange!30]
\tikzstyle{container}=[rectangle, rounded corners = 0.5cm, draw=black, fill=green!30, text centered, text depth = 3cm, minimum width=4.5cm, minimum height=3.7cm]
\tikzstyle{container2}=[rectangle, rounded corners = 0.5cm, draw=black, fill=green!30, text centered, text depth = 8.7cm, minimum width=7.5cm, minimum height=9.7cm]
\tikzstyle{decision}=[diamond, draw=black, fill=yellow!30, text centered]

\newcommand{\keywords}[1]{\par\addvspace\baselineskip
\noindent\keywordname\enspace\ignorespaces#1}

\pagestyle{headings}
% !TeX spellcheck = en_US 

\addtolength{\textfloatsep}{-15px}
\addtolength{\intextsep}{-5px}

\begin{document}

\mainmatter  % start of an individual contribution

% first the title is needed
\title{Frequency Modulation with Feedback in Granular Synthesis}

% a short form should be given in case it is too long for the running head
\titlerunning{FM with Feedback in GS}


% TO GARANTEE THE DOUBLE BLIND REVIEW PROCESS, PLEASE
% KEEP THESE GENERIC AUTHOR NAMES AND INSTITUTIONS

\author{Øyvind Brandtsegg, Victor Lazzarini}
%

\institute{Norwegian University of Science and Technology, Maynooth University \\ \email{oyvind.brandtsegg@ntnu.no, victor.lazzarini@mu.ie}}



\maketitle

\begin{abstract}

The paper investigates audio synthesis with frequency modulation feedback in granular synthesis, comparing it with regular FM feedback. 

\keywords{...}
\end{abstract}


\section{Introduction}
FM synthesis is well known ..., FM feedback has been explored by (Lazzarini 2024),...here we will look at FM with feedback within granular synthesis as it allows some new means of pitch stabilisation, poses some new problems, and enables some exciting new sonic extensions to both FM and granular synthesis domains.

\section{Running the provided code examples}
The code examples for this paper can be found in a github repository at \url{https://github.com/Oeyvind/partikkel_fm}. There are Csound orchestra files for FM with granular synthesis and with regular oscillators. To compare the two techniques, it is useful to run both with the same set of parameters (adding only a few extra parameters for granular synthesis), and then compare the generated sound files. The repo contains python files to write score files and render sound with both techniques. This will also display spectrograms for the two generated sound files. The Python files contain default parameter settings as a starting point, and allow parameter modification via command line arguments. For example:
\begin{verbatim}
python generate_and_compare.py soundfilename cps=200
\end{verbatim}
will modify the fundamental frequency (cps) by setting it to 200Hz, then render sound with both synthesis techniques and display spectrograms for both generated sound files. 

\section{Conclusion}
...

%\pagebreak
\begin{thebibliography}{4}


\bibitem{Brandtsegg-particle} Brandtsegg, Ø. and Saue, S. and Johansen, T. (2011) \emph{Particle synthesis–a unified model for granular synthesis}. Proceedings of the 2011 Linux Audio Conference. \url{http://lac.linuxaudio.org/2011/papers/39.pdf}

\bibitem{Cabbage-url} Walsh, R. \emph{Cabbage - A framework for audio software development.} \url{https://cabbageaudio.com}

\bibitem{Lazzarini-2016} Lazzarini, V. et al. (2016). \emph{Csound: A Sound and Music Computing System.} Springer.

\bibitem{Lazzarini-2024} Lazzarini, V. and Timoney, J. \emph{Theory and Practice of Higher-Order Frequency Modulation
Synthesis}


\end{thebibliography}



\end{document}
